\documentclass[12pt,a4paper]{article}
\sloppy
\usepackage[utf8]{inputenc}
\usepackage[english]{babel}
\usepackage[T1]{fontenc}
\usepackage{natbib}
\usepackage{amsmath}
\usepackage{lscape}
\usepackage{amsfonts}
\usepackage{amssymb}
\usepackage{graphicx}
\usepackage[left=2.5cm,right=2.5cm,top=2.5cm,bottom=2.5cm]{geometry}
\usepackage{cancel}
\usepackage{tikz}
\usepackage{float}
\usepackage{multirow}
\usepackage[breaklinks=true]{hyperref}
\usepackage{framed}
\usepackage{stmaryrd}

%%%%%%%%%%%%%%%% Informacion de documento %%%%%%%%%%%%%%%%%%%%
\author{Mansilla, Kevin Gaston\footnote{\href{mailto:kevin.mansilla@mi.unc.edu.ar}{kevin.mansilla@mi.unc.edu.ar}}}
\title{Laboratorio Lenguajes y Compiladores}
\date{\today}
%%%%%%%%%%%%%%%%%%%%%%%%%%%%%%%%%%%%%%%%%%%%%%%%%%%%%%%%%%%%%%

\newcommand{\R}{\mathbb{R}}
\newcommand{\N}{\mathbb{N}}
\newcommand{\Q}{\mathbb{Q}}
\newcommand{\Z}{\mathbb{Z}}
\newcommand{\K}{\mathbb{K}}
\newcommand{\<}{\leqslant}
\renewcommand{\>}{\geqslant}
\providecommand{\abs}[1]{\left|#1\right|}
\providecommand{\norma}[1]{\left||#1\right||}
\providecommand{\indexa}{\triangleright}
\providecommand{\parent}[1]{\left(#1\right)}
\providecommand{\corchet}[1]{\left[#1\right]}
\providecommand{\prop}[1]{\langle #1\rangle}
\providecommand{\llaves}[1]{\left\lbrace #1 \right\rbrace}
\newtheorem{teorema}{Teorema}
\newtheorem{corolario}{Corolario}
\newtheorem{definition}{Definición}
\newtheorem{ejemplo}{Ejemplo}
\newtheorem{nota}{Nota}
\newtheorem{ejercicio}{Ejercicio}
\renewcommand{\tablename}{Tabla}
\renewcommand\refname{Bibliograf{\'i}a}
\renewcommand{\listtablename}{Índice de tablas}
\renewcommand{\figurename}{Gráfico}
\renewcommand{\thesection}{\Roman{section}}

%%
%\documentclass[12pt,a4paper]{article}
%\sloppy
\usepackage[utf8]{inputenc}
\usepackage[english]{babel}
\usepackage[T1]{fontenc}
\usepackage{natbib}
\usepackage{amsmath}
\usepackage{lscape}
\usepackage{amsfonts}
\usepackage{amssymb}
\usepackage{graphicx}
\usepackage[left=2.5cm,right=2.5cm,top=2.5cm,bottom=2.5cm]{geometry}
\usepackage{cancel}
\usepackage{tikz}
\usepackage{float}
\usepackage{multirow}
\usepackage[breaklinks=true]{hyperref}
\usepackage{framed}
\usepackage{stmaryrd}

%%%%%%%%%%%%%%%% Informacion de documento %%%%%%%%%%%%%%%%%%%%
\author{Mansilla, Kevin Gaston\footnote{\href{mailto:kevin.mansilla@mi.unc.edu.ar}{kevin.mansilla@mi.unc.edu.ar}}}
\title{Laboratorio Lenguajes y Compiladores}
\date{\today}
%%%%%%%%%%%%%%%%%%%%%%%%%%%%%%%%%%%%%%%%%%%%%%%%%%%%%%%%%%%%%%

\newcommand{\R}{\mathbb{R}}
\newcommand{\N}{\mathbb{N}}
\newcommand{\Q}{\mathbb{Q}}
\newcommand{\Z}{\mathbb{Z}}
\newcommand{\K}{\mathbb{K}}
\newcommand{\<}{\leqslant}
\renewcommand{\>}{\geqslant}
\providecommand{\abs}[1]{\left|#1\right|}
\providecommand{\norma}[1]{\left||#1\right||}
\providecommand{\indexa}{\triangleright}
\providecommand{\parent}[1]{\left(#1\right)}
\providecommand{\corchet}[1]{\left[#1\right]}
\providecommand{\prop}[1]{\langle #1\rangle}
\providecommand{\llaves}[1]{\left\lbrace #1 \right\rbrace}
\newtheorem{teorema}{Teorema}
\newtheorem{corolario}{Corolario}
\newtheorem{definition}{Definición}
\newtheorem{ejemplo}{Ejemplo}
\newtheorem{nota}{Nota}
\newtheorem{ejercicio}{Ejercicio}
\renewcommand{\tablename}{Tabla}
\renewcommand\refname{Bibliograf{\'i}a}
\renewcommand{\listtablename}{Índice de tablas}
\renewcommand{\figurename}{Gráfico}
\renewcommand{\thesection}{\Roman{section}}

%%
%\documentclass[12pt,a4paper]{article}
%\sloppy
\usepackage[utf8]{inputenc}
\usepackage[english]{babel}
\usepackage[T1]{fontenc}
\usepackage{natbib}
\usepackage{amsmath}
\usepackage{lscape}
\usepackage{amsfonts}
\usepackage{amssymb}
\usepackage{graphicx}
\usepackage[left=2.5cm,right=2.5cm,top=2.5cm,bottom=2.5cm]{geometry}
\usepackage{cancel}
\usepackage{tikz}
\usepackage{float}
\usepackage{multirow}
\usepackage[breaklinks=true]{hyperref}
\usepackage{framed}
\usepackage{stmaryrd}

%%%%%%%%%%%%%%%% Informacion de documento %%%%%%%%%%%%%%%%%%%%
\author{Mansilla, Kevin Gaston\footnote{\href{mailto:kevin.mansilla@mi.unc.edu.ar}{kevin.mansilla@mi.unc.edu.ar}}}
\title{Laboratorio Lenguajes y Compiladores}
\date{\today}
%%%%%%%%%%%%%%%%%%%%%%%%%%%%%%%%%%%%%%%%%%%%%%%%%%%%%%%%%%%%%%

\newcommand{\R}{\mathbb{R}}
\newcommand{\N}{\mathbb{N}}
\newcommand{\Q}{\mathbb{Q}}
\newcommand{\Z}{\mathbb{Z}}
\newcommand{\K}{\mathbb{K}}
\newcommand{\<}{\leqslant}
\renewcommand{\>}{\geqslant}
\providecommand{\abs}[1]{\left|#1\right|}
\providecommand{\norma}[1]{\left||#1\right||}
\providecommand{\indexa}{\triangleright}
\providecommand{\parent}[1]{\left(#1\right)}
\providecommand{\corchet}[1]{\left[#1\right]}
\providecommand{\prop}[1]{\langle #1\rangle}
\providecommand{\llaves}[1]{\left\lbrace #1 \right\rbrace}
\newtheorem{teorema}{Teorema}
\newtheorem{corolario}{Corolario}
\newtheorem{definition}{Definición}
\newtheorem{ejemplo}{Ejemplo}
\newtheorem{nota}{Nota}
\newtheorem{ejercicio}{Ejercicio}
\renewcommand{\tablename}{Tabla}
\renewcommand\refname{Bibliograf{\'i}a}
\renewcommand{\listtablename}{Índice de tablas}
\renewcommand{\figurename}{Gráfico}
\renewcommand{\thesection}{\Roman{section}}

%%
%\documentclass[12pt,a4paper]{article}
%\input{packages.tex}
% \begin{document}
%\maketitle{}


% \end{document}
% \begin{document}
%\maketitle{}


% \end{document}
% \begin{document}
%\maketitle{}


% \end{document}
\begin{document}
\maketitle{}

Implementación de la semántica denotacional para el lenguaje imperativo simple 
con fallas (LIS + Fallas).

\section{Semántica del lenguaje}
Se agregan excepciones al lenguaje imperativo simple.
$$\langle comm \rangle ::= fail \mid catchin \langle comm \rangle with \langle comm \rangle$$

Sea $\Sigma^{'} = \Sigma \cup \{ abort\}$ con orden discreto. La guncion semántica ahora es:
\begin{align*}
\llbracket\_\,\rrbracket \in \langle comm \rangle \to \Sigma \to \Sigma^{'}_{\bot}
\end{align*}

Las ecuaciones semánticas son las siguiente:
\begin{align*}
\llbracket Skip \rrbracket_{\sigma} & = \sigma \\
\llbracket Fail\rrbracket_{\sigma} &= \langle abort, \sigma \rangle\\
\llbracket v := e\rrbracket_{\sigma} &= [\sigma\mid v : \llbracket e\rrbracket_{\sigma} ]\\
\llbracket if\,\, b\,\, then\,\, c_{0}\,\, else\,\, c_{1} \rrbracket_{\sigma} &= \begin{cases}
  \llbracket c_{0}\rrbracket_{\sigma} & \text{si}\,\, \llbracket b\rrbracket_{\sigma}\\
  \llbracket c_{1}\rrbracket_{\sigma} & \text{si}\,\, \neg \llbracket b\rrbracket_{\sigma}
\end{cases}\\
\end{align*}


Operadores de transferencia de control. Dada $f \in \Sigma \to \Sigma^{'}_{\bot}$,
denotamos por $f_{*}$ la siguiente funcion
\begin{align*}
f_{*} & \in \Sigma^{'}_{\bot} \to \Sigma^{'}_{\bot}\\
f_{*}x &= \begin{cases}
  f \sigma & \text{si}\,\, x = \sigma \in \Sigma\\
  x & \text{si no}
\end{cases}
\end{align*}

En este caso, la presencia de una situación obortiva determina que no se 
transfiere el control a $f$ Servira para describir el significado de $c_{0};c_{1}$
ya que si ocurre una situacion de excepción al ejecutar $c_{0}$, en control no es 
transferido a $c_{1}$.

Por otro lado, dado $f\in \Sigma \to \Sigma^{'}_{\bot}$, denotamos $f_{+}$ a la sigueinte extension de $f$ a $\Sigma^{'}_{\bot}$:
\begin{align*}
f_{+} & \in \Sigma^{'}_{\bot} \to \Sigma^{'}_{\bot}\\
f_{*}x &= \begin{cases}
  f \sigma & \text{si}\,\, x = \langle abort, \sigma \rangle \in \{ abort\}\times \Sigma\\
  x & \text{si no}
\end{cases}
\end{align*}

En una clara dualidad con la definición de $f_{*}$, la definición de $f_{+}$ 
determina lo contrario: se transfiere el control a $f$ sólo en caso de excepción. 
Esto corresponderá a $catchin\,\, c\,\, with\,\, c'$.

Finalmente, dada $f\in \Sigma \to \Sigma$, denotamos por $f_{\dagger}$ la siguiente
extension de $f$ a $\Sigma^{'}_{\bot}$:
\begin{align*}
f_{\dagger} & \in \Sigma^{'}_{\bot} \to \Sigma^{'}_{\bot}\\
f_{\dagger}x &= \begin{cases}
  \langle abort, f\sigma \rangle & \text{si}\,\, x = \langle abort, \sigma \rangle \\
  f x & \text{si}\,\, x \in \Sigma\\
  \bot & \text{si no}
\end{cases}
\end{align*}


Nose que aui hay una transferencia de control a $f$ en cualquier situacion (abortiva o no). Servirá para restaurar el valor de las variables locales. Entonces las 
demas funcione semanticas
\begin{align*}
\llbracket c_{0};c_{1} \rrbracket_{\sigma} &= \llbracket c_{1}\rrbracket_{*}(\llbracket c_{0}\rrbracket_{\sigma})\\
\llbracket catchin \,\,c_{0}\,\, with \,\,c_{1} \rrbracket_{\sigma} &= \llbracket c_{1}\rrbracket_{+}(\llbracket c_{0}\rrbracket_{\sigma})\\
\llbracket newvar\,\, v := e\,\, in\,\, c \rrbracket_{\sigma} &= (\lambda\sigma^{'} \in \Sigma. 
[\sigma^{'} \mid v : \sigma v])_{\dagger} (\llbracket c\rrbracket[\sigma \mid v : \llbracket e\rrbracket_{\sigma} ])\\
\llbracket while\,\,b\,\,do\,\,c\rrbracket &= \sqcup_{i=0}^{\infty} F^{i} \bot_{\Sigma \to \Sigma_{\bot}} \\
donde \\
F w \sigma &= \begin{cases}
  w_{*}(\llbracket c\rrbracket_{\sigma}) & \text{si}\,\, \llbracket b\rrbracket_{\sigma}\\
  \sigma & \text{si}\,\, \neg \llbracket b\rrbracket_{\sigma}
  \end{cases}
\end{align*}

\end{document}